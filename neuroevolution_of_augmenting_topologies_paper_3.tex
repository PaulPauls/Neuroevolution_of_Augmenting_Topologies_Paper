% ######################################################################################################################

\documentclass[journal, a4paper]{IEEEtran}

\usepackage{graphicx}       % For graphics, photos, etc
\usepackage{hyperref}       % For URL and href
\usepackage{amsmath}        % For advanced mathematical formatting and symbols
\usepackage{blindtext}      % For placeholder text
\usepackage{listings}       % For code listings
\usepackage{color}          % For color
\usepackage{draftwatermark} % For watermark

\graphicspath{{./illustrations/}}

\definecolor{green}{rgb}{0, 0.66, 0}
\definecolor{red}{rgb}{1, 0, 0}
\definecolor{gray}{rgb}{0.5, 0.5, 0.5}
\definecolor{orange}{rgb}{1, 0.66, 0}
\definecolor{codebg}{rgb}{0.97, 0.97, 0.97}

\newcommand{\customincludegraphics}[2]{
}
 
\lstdefinestyle{c-style}{
  language={[ANSI]C},
  frame=single,
  backgroundcolor=\color{codebg},
  commentstyle=\itshape\color{green},
  keywordstyle=\color{blue},
  numberstyle=\tiny\color{gray},
  stringstyle=\color{orange},
  basicstyle=\fontsize{7}{7}\ttfamily,
  breakatwhitespace=false,
  breaklines=true,
  captionpos=b,
  keepspaces=true,
  numbers=left,
  numbersep=5pt,
  showspaces=false,
  showstringspaces=false,
  showtabs=false,
  tabsize=2
}

% ######################################################################################################################

\begin{document}

\title{On Overview of Neuroevolution and NEAT}
\author{Paul Pauls\\
        Advisor: Michael Adam}
\markboth{Neuroevolution of Augmenting Topologies}{}
\maketitle

% Place small Watermark indicating that this is currently a draft in background
\SetWatermarkText{DRAFT}
\SetWatermarkScale{0.5}

% While Paper is in development shall I include this table of contents as a quick overview
\tableofcontents

\begin{abstract}
    \blindtext
\end{abstract}

% ######################################################################################################################

\section{Introduction}

\IEEEPARstart{T}{his} shall be my introduction. And this shall be my citation \cite{cite01}.
\blindtext

% CHECK: HAVE I REMOVED ALL PERSONAL ADRESSING (I, MY, etc) FROM THIS PAPER AND IS IT WRITTEN SCIENTIFICALLY/NEUTRAL?
% CHECK: WORK THROUGH PRECEDING VERSIONS OF THIS PAPER AND ADD IMPORTANT INFOS/PHRASINGS THAT I HAD BEFORE



% ######################################################################################################################

\section{Neuroevolution and Evolutionary Algorithms}

Neuroevolution is a form of evolutionary algorithm that generates specific artificial neural networks (short form: ANN) through modification of its parameters, topology and rules in order to maximize the ANN's accuracy or fitness score. The neuroevolution algorithm seeks to modify the ANN in an evolutionary process similar to the Darwinian process that produced human brains and its process-summarizing maxim 'survival of the fittest'. First methods using neuroevolution can be traced back to the 1980s and 1990s \href{https://www.inovex.de/blog/neuroevolution/}{[cite]}, while the first evolutionary algorithms were conceived in the 1950s by Alan Turing and Nils Barricelli. \href{https://en.wikipedia.org/wiki/Genetic_algorithm#cite_note-mind.oxfordjournals.org-33}{[cite]}

To better characterize neuroevolution is it best to first roughly categorize it, whereupon the following sections describe the categories in detail. A \textit{neuroevolution algorithm} is a \textit{genetic algorithm}, whose search-space (or genotypes) consist only of artificial neural networks. A \textit{genetic algorithm} in turn is a \textit{evolutionary algorithm} that evolves genotypes - genetically encoded representations of the actual solution (phenotype) - through mutation, recombination and selection.



\subsection{Evolutionary Algorithms}

A evolutionary algorithm (short form: EA) is defined as 'a generic population-based and meta-heuristically optimized algorithmic solution to an applied problem' \href{https://en.wikipedia.org/wiki/Evolutionary_algorithm}{[cite]}. When breaking this complex definition down into simpler terms, is the first thing that should be clarified about EAs the fact that they are no algorithmic solution to the applied problem in and of themselves per se, but rather that they are meta-algorithms that create another optimized algorithm, which then solves the applied problem. An evolutionary algorithm therefore encodes a method of how to come up with the best solution to an algorithmic problem.

Evolutionary algorithms set out to finding this best algorithmic solution through a \textit{population-based} method. This means that EAs manage an arbitrary variety of algorithmic solutions - all of which differ and solve the applied problem with various grades of accuracy or fitness scoring. Each of these algorithmic solutions is called a \textit{member} of the evolutionary algorithms' population and is potentially the best algorithmic solution - the best member in other words - that is eventually returned. To determine the best member of the population does the evolutionary algorithm assign each member a \textit{fitness score} after it is created. In the context of neuroevolution for example is this fitness score calculated by judging the accuracy of the artificial neural network or it is calculated by the fitness function in case of an environment embedded agent.

However, the question remains how the members of the population are created in a sensible way so that they may represent a reasonable algorithmic solution to the problem and eventually an optimized one. This is the point at which the EAs' aspect of \textit{meta-heuristic optimization} comes into play. Each new member in a population is conceived by recombining and/or mutating a single or multiple existing members of the population. Presuming that additionally the single or multiple existing members that are recombined and/or mutated to create the new member are chosen to be the highest fitness scoring members (a.k.a. the \textit{fittest} members), does the process constitute an optimization procedure resembling Darwinian evolution. Therefore can be said that evolutionary algorithms breed increasingly optimized algorithmic solution through evolutionary intercombination of existing algorithmic solutions - hence representing the mentioned optimization process. 

The possible methods of intercombination between existing members are inspired by biological evolution, such as mutation, recombination and selection. For the sake of brevity will only the subsequent chapter \ref{ch:neuroevolution} explain those intercombination methods in detail and how they apply to neuroevolution algorithms. The mentioned optimization process achieved through these intercombination methods is considered \textit{meta-heuristic} because the optimization process is possible with incomplete or imperfect information or limited computational capacity. Thus even when the feedback - meaning the ability to determine a sensible fitness score through accuracy measurement or similar - of the applied problem is limited or sparse, is fruitful traversal across the search-space through e.g. a lucky mutation feasible.

Finally can be said that the evolutionary algorithms population-based methodology is considered \textit{generic} because it does not dictate how the members of the population are encoded. Though an evolutionary algorithm needs to eventually return an algorithmic solution to the applied problem, does it not dictate that the members are algorithms in memory, nor that they even need to be in algorithmic form when they are given a fitness score. This is were genetic algorithms come into play.



\subsection{Genetic Algorithms}

Genetic algorithms (short form: GAs) are evolutionary algorithms whose members are not saved as algorithms in memory, but as genetic-like encodings which can then be translated into algorithms by a user-specified component of the genetic algorithm. The genetic-like encoded member in a GA is called \textit{genotype}, whereas its corresponding translated algorithm is called \textit{phenotype} To give an example, a genetic algorithm trying to find the best search algorithm would not represent a member such as the Quicksort algorithm itself in memory, but as a series of four different characters which would then in turn translate into the Quicksort algorithm. Its genotype could be a character-string such as "ACGCUG...", while its phenotype would be the explicit search algorithm in code.

The defining advantage of genetic algorithms by representing members as a genetic-like encoding instead of an algorithm in memory is that intercombination methods like mutation and recombination - central aspects of evolutionary algorithms - are vastly easier and faster on relatively simple genetic encodings than on complex specified algorithms. Figure \ref{ill:comp_conv}, though actually illustrating the 'Competing Conventions Problem', also illustrates well how such a complex algorithm as an artificial neural network is significantly simpler represented as a genetic encoding and how such a genetic encoding can be vastly easier recombined than the algorithm itself.



\subsection{Neuroevolution}\label{ch:neuroevolution}

To come full circle is now a proper definition of Neuroevolution possible. Neuroevolution is the - possibly boundless - process in which by the means of a genetic algorithm its population of artificial neural networks is increasingly optimized in order to maximize the accuracy or fitness of the best ANN. Neuroevolution does so by continuously improving the members in its population through intercombination methods like mutation, recombination and selection.

\begin{figure}
    \includegraphics[width=0.45\textwidth]{{Binary_ANN_Encoding_-_yao99}}
    \caption{Illustration of a binary encoding of an ANN. (Source: \cite{yao99})}
    \label{ill:binary_encoding}
\end{figure}

The intercombinatory method of \textit{mutation} in the context of Neuroevolution means that the genotype of a chosen member is in some way modified by adding to, changing or removing from the genotype representation. The manner and probability in which this modification takes place is completely up to the implementation specifics of the respective neuroevolution algorithm. To give an example mutation for an ANN, is it possible to imagine a binary encoded genotype as seen in figure \ref{ill:binary_encoding}, which then has some bits flipped, added or removed possibly resulting in a new node or connection in its corresponding phenotype.

The intercombinatory method of \textit{recombination} in the context of Neuroevolution means that the genotype of two or more arbitrary (though most often high performing) members are combined, forming a new member. This is done in the hopes that those parts of the genotype that encode member-distinct features combine into the newly created genotype and encode an ANN that performs even better than both \textit{parent}-members in isolation. An example of such a recombination, though an impaired one as the figure actually represents the flawed process of the 'Competing Conventions Problem', can be seen in figure \ref{ill:comp_conv}. Again is the manner and probability in which this modification is performed completely up to the implementation specifics of the respective neuroevolution algorithm.

Lastly is the intercombinatory method of \textit{selection}. As previously defined does the process of neuroevolution work on populations; these however are often of fixed size in most neuroevolution algorithms. The purpose of this restriction is to force the neuroevolution process to remove low performing members from the population - and therefore the gene pool - from which possible parents for intercombination are chosen, by only allowing a limited number of members to exist. Once all members of the population have been assigned a fitness score, is this current state of the population considered a specific \textit{generation}. The method of \textit{selection} then removes certain members of the population while the intercombinatory methods of mutation and recombination add new members to the population and the whole population is evaluated again, marking the start of the next generation. The method of selection is also applicable in the case of an unrestricted population size, e.g. by removing members that score too far below the current best member.

All those Details of the neuroevolution process, e.g. how the encoding scheme specifies genotypes and their translation into the phenotype ANNs, how the initial population is created, which members are chosen as parents for intercombinatory methods or simply how exactly the intercombinatory methods are performed are all left to the specific neuroevolution algorithm.



\subsection{Landmark Research in Neuroevolution}

% Neurevolution first gained significatn traction in the reserach commmunity with NEAT, as it was the first to address fundamental shortcomings from which earlier research suffered (e.g. ... https://en.wikipedia.org/wiki/Neuroevolution)

%##################
% From OReilly Blog: In 1999, I started to think seriously about neuroevolution as an early Ph.D. student at the University of Texas at Austin in the research group of Professor Risto Miikkulainen, where several important fixed-topology algorithms had already been invented (such as the SANE and ESP algorithms from David Moriarty and Faustino Gomez, both in collaboration with Risto Miikkulainen). These algorithms were already quite clever, investigating the idea that the individual neurons of a network might be evolved in their own subpopulations to cooperate with other neurons, and then grouped with neurons from other subpopulations to form full working networks, an idea generally known as “cooperative coevolution.”

%Arriving on the scene in the midst of these developments, I was enchanted with the idea that the evolution of brains could happen inside a computer. I appreciated the importance of evolving ANNs to solve problems, but my real passion lay with the idea of increasing complexity over evolution. I wanted to build an algorithm that would enable ANNs to explode in complexity inside the computer, echoing the evolution of brains in nature. For some reason, even though existing TWEANN algorithms could change the architecture of evolving ANNs over time, they seemed to lack the strong propensity toward increasing complexity that I hoped to capture.

%I spent a lot of time studying existing TWEANN algorithms and thinking about why they might not be working as well as they could. Some problems were well known, such as the competing conventions problem, which refers to the fact that it is hard to combine two parent ANNs to create an offspring (an operation called “crossover”). More specifically, different networks might express the same functionality with different combinations of connections weights, making it hard to know how to combine them as parents.
%########

% Mention move from non-TWEANNs to TWEANNs and how nowadays SGD like backpropagation is generally preferred for weight adaptation, though currenlty there is a consesus in the neuroevolution community thata combined approach is the most promising (Evolce and Transfer weights roughly, finetun with SGD)

NEAT (2002)
% Neat and its variants (rtNeat, odNEAT, etc) from which only HyperNEAT survived being an actual game changer and again being developed by Stanley

HyperNEAT (2009)

ES-HyperNEAT (2011)

Novelty Search (2011)
% Novelty search is another invention that abandons the goal to maximise a fitness function. Instead, it rewards individuals that behave differently. This change often leads to much better solutions in the end because novel solutions perform badly in the beginning and need time to fully manifest their potential.

Coevolution DeepNEAT (2017)
% Miikkulainen, Liang - Evolving Deep Neural Networks
% See innovex summary

real17 (2017)
% Include progress picture of the population of image classifier over time

EvoCNN (2017)
% Sun, Xue - Evolving Deep Convolutional Neural Networks for Image classification
% see innovex summary


Uber Research (End 2017 / more in 2018)

liu18 (2018)

real19 (2019)

% Include the comparison picture of NE vs RL vs RS






% ######################################################################################################################

\section{NeuroEvolution of Augmenting Topologies (NEAT)}

% Read papers sta02_1, sta02_2, sta04 again

<Section Introduction>

Neuroevolution of Augmenting Topologies (short form: NEAT) was first introduced in the paper "Evolving Neural Networks through Augmenting Topologies" by Kenneth O.Stanley and Risto Miikkulainen in the year 2002. [cite] It was finalized and shown to be superior to any preceding neuroevolution algorithm in Stanley's PhD thesis "Efficient Evolution of Neural Networks through Complexification" in 2004. [cite]



At time of envisioning of NEAT was Neuroevolution most promising learning approach. Still is powerful today (see rea17/19)
"NE is a promising approach to learning behavioral policies and finds solutions faster than leading RL methods on many benchmark tasks (Gomez 2003; Moriarty and Miikkulainen 1997)" \cite{sta04}


"In highly complex domains the heuristics for determining the appropriate size are not very useful, and it becomes increasingly difficult to solve such domains with fixed-length encodings." \cite{sta04}

[See all notes write down about stanleys PhD thesis]




\subsection{Key Aspects of NEAT and Differences to Preceding Neuroevolution}
% Solving Competing Convetions, Speciation, Historical Markings, Minimal Initial Pop, See Key Elements identified through ablation (http://nn.cs.utexas.edu/downloads/papers/stanley.cec02.pdf)

\begin{figure}
    \includegraphics[width=0.45\textwidth]{{Competing_Conventions_Crossover_-_sta02_2}}
    \caption{Illustration of the 'Competing Conventions Problem'. (Source: \cite{sta02_2})}
    \label{ill:comp_conv}
\end{figure}



\subsection{Performance of NEAT}
% Include traditional NEAT performance but also NEAT performance in more current examples

\subsection{Variants and Advancements of NEAT}
% Follow Up Research and Variants (HyperNeat, ES-HyperNeat, Novelty Search, but also small variations of NEAT such as aging NEAT or rtNEAT)
% Mention non-mating variants of neuro evolution algorithms /neat (such as the current research in rea17/19). See chapter 2.3.2 'non-mating' in [sta04] for an overview

\subsubsection{<Variant 1>}
\subsubsection{<Variant 2>}
\subsubsection{<Variant 3>}



% ######################################################################################################################

\section{Practical Applications of NEAT}
% Go especially into detail what Stanley considered great NEAT applications in his reddit AMA
% Introduce my own code accompanying this paper

\subsection{<Application 1>}
\subsection{<Application 2>}
\subsection{<Application 3>}




% ######################################################################################################################

\section{Conclusion}

\blindtext



% ######################################################################################################################

\section{Acknowledgments}

I wish to thank Rezsa Farahani, from the Google Brain team, and Michael Adam, from the Technical University Munich, for their support and valuable input.



% ######################################################################################################################

\begin{thebibliography}{5}

  \bibitem{yao99}
    Yao - Evolving Artificial Neural Networks; 1999;
    \url{http://avellano.fis.usal.es/~lalonso/compt_soft/articulos/yao99evolving.pdf}

  \bibitem{sta02_1}
    Stanley, Miikkulainen - Efficient Evolution of Neural Network Topologies; 2002;
    \url{http://nn.cs.utexas.edu/downloads/papers/stanley.cec02.pdf}

  \bibitem{sta02_2}
    Stanley, Miikkulainen - Evolving Neural Networks through Augmented Topologies; 2002;
    \url{http://nn.cs.utexas.edu/downloads/papers/stanley.ec02.pdf}

  \bibitem{gea03}
    Geard, Wiles - Structure and Dynamics of a Gene Network Model Incorporating Small RNAs; Dec 2003;
    \url{https://ieeexplore.ieee.org/document/1299575}

  \bibitem{sta04}
    Stanley - Efficient Evolution of Neural Networks through Complexification; Aug 2004;
    \url{http://nn.cs.utexas.edu/downloads/papers/stanley.phd04.pdf}

  \bibitem{rei07}
    Reisinger, Miikkulainen - Acquiring Evolvability through Adaptive Representations; Jul 2007;
    \url{http://nn.cs.utexas.edu/downloads/papers/reisinger.gecco07.pdf}

  \bibitem{mat07}
    Mattiussi, Duerr, et al - Center of Mass Encoding: A self-adaptive representation with adjustable redundancy for real-valued parameters; Jul 2007;
    \url{https://infoscience.epfl.ch/record/101405}

  \bibitem{flo08}
    Floreano, Duerr, et al - Neuroevolution: From Architectures to Learning; Jan 2008;
    \url{http://citeseerx.ist.psu.edu/viewdoc/summary?doi=10.1.1.182.1567}

  \bibitem{mat08}
    Mattiussi, Marbach, et al - The Age of Analog Networks; Sep 2008;
    \url{https://www.aaai.org/ojs/index.php/aimagazine/article/view/2156}

  \bibitem{sta09}
    Stanley, D’Ambrosio, et al - A Hypercube-Based Indirect Encoding for Evolving Large-Scale Neural Networks; 2009;
    \url{http://axon.cs.byu.edu/~dan/778/papers/NeuroEvolution/stanley3**.pdf}

  \bibitem{ris11}
    Risi, Stanley - Enhancing ES-HyperNEAT to Evolve More Complex Regular Neural Networks; Jul 2011;
    \url{http://citeseerx.ist.psu.edu/viewdoc/summary?doi=10.1.1.365.4332}

  \bibitem{leh11}
    Lehman, Stanley - Novelty Search and the Problem with Objectives; Oct 2011;
    \url{https://www.cs.ucf.edu/eplex/papers/lehman_gptp11.pdf}

  \bibitem{woe12}
    Woergoetter, Porr - Scholarpedia Article on 'Reinforcement Learning'; Sep 2012;
    \url{http://www.scholarpedia.org/article/Reinforcement_learning}

  \bibitem{hol12}
    Holland - Scholarpedia Article on 'Genetic Algorithms'; Oct 2012;
    \url{http://www.scholarpedia.org/article/Genetic_algorithms}

  \bibitem{fog13}
    Fogel, Fogel, et al - Scholarpedia Article on 'Evolutionary Programming'; Oct 2013;
    \url{http://www.scholarpedia.org/article/Evolutionary_programming}

  \bibitem{leh13}
    Lehman, Miikkulainen - Scholarpedia Article on 'Neuroevolution'; Oct 2013;
    \url{http://www.scholarpedia.org/article/Neuroevolution}

  \bibitem{pas14}
    Pascanu, Ganguli, et al - On the Saddle Point for Non-Convex Optimization; May 2014;
    \url{https://www.researchgate.net/publication/262452520}

  \bibitem{kim15}
    Kim, Rigazio - Deep Clustered Convolutional Kernels; Mar 2015;
    \url{https://arxiv.org/abs/1503.01824}

  \bibitem{fer16}
    Fernando, Banarse, et al - Convolution by Evolution; Jun 2016;
    \url{https://arxiv.org/abs/1606.02580}

  \bibitem{mii17}
    Miikkulainen, Liang, et al - Evolving Deep Neural Networks; Mar 2017;
    \url{https://arxiv.org/abs/1703.00548}

  \bibitem{xie17}
    Xie, Yuille - Genetic CNN; Mar 2017;
    \url{https://arxiv.org/abs/1703.01513}

  \bibitem{neg17}
    Negrinho, Gordon - DeepArchitect: Automatically Designing and Training Deep Architectures; Apr 2017;
    \url{https://arxiv.org/abs/1704.08792}

  \bibitem{rea17}
    Real, Moore, et al - Large-scale Evolution of Image Classifiers; Jun 2017;
    \url{https://arxiv.org/abs/1703.01041}

  \bibitem{sta17_neuroevolution_overview}
    Stanley - Neuroevolution: A Different Kind of Deep Learning; Jul 2017;
    \url{https://www.oreilly.com/ideas/neuroevolution-a-different-kind-of-deep-learning}

  \bibitem{bro17}
    Brock, Lim, et al - SMASH: One-Shot Model Architecture Search through HyperNetworks; Aug 2017;
    \url{https://arxiv.org/abs/1708.05344}

  \bibitem{sal17}
    Salimans, Ho - Evolution Strategies as a Scalable Alternative to Reinforcement Learning; Sep 2017;
    \url{https://arxiv.org/abs/1703.03864}

  \bibitem{jad17}
    Jaderberg, Dalibard, et al - Population Based Training of Neural Networks; Nov 2017;
    \url{https://arxiv.org/abs/1711.09846}

  \bibitem{zha17}
    Zhang, Clune, et al - On the Relationship Between the OpenAI Evolution Strategy and Stochastic Gradient Descent; Dec 2017;
    \url{https://arxiv.org/abs/1712.06564}

  \bibitem{sta17_deep_neuroevolution}
    Stanley, Clune - Welcoming the Era of Deep Neuroevolution; Dec 2017;
    \url{https://eng.uber.com/deep-neuroevolution/}

  \bibitem{liu18}
    Liu, Simonyan, et al - Hierarchical Representation for Efficient Architecture Search; Feb 2018;
    \url{https://arxiv.org/abs/1711.00436}

  \bibitem{suc18_accelerating_deep_neuroevolution}
    Such, Stanley, et al - Accelerating Deep Neuroevolution: Train Atari in Hours on a Single Personal Computer; Apr 2018;
    \url{https://eng.uber.com/accelerated-neuroevolution/}

  \bibitem{suc18_introduction_deep_neuroevolution}
    Such, Madhavan, et al - Deep Neuroevolution: Genetic Algorithms Are a Competitive Alternative for Training Deep Neural Networks for Reinforcement Learning; Apr 2018;
    \url{https://arxiv.org/abs/1712.06567}

  \bibitem{zop18}
    Zoph, Vasudevan, et al - Learning Transferable Architectures or Scalable Image Recognition; Apr 2018;
    \url{https://arxiv.org/abs/1707.07012}

  \bibitem{leh18_evolution_strategy}
    Lehman, Chen, et al - ES Is More Than Just a Traditional Finite-Difference Approximator; May 2018;
    \url{https://arxiv.org/abs/1712.06568}

  \bibitem{leh18_safe_mutations}
    Lehman, Chen, et al - Safe Mutations for Deep and Recurrent Neural Networks through Output Gradients; May 2018;
    \url{https://arxiv.org/abs/1712.06563}

  \bibitem{zho18}
    Zhong, Yan, et al - Practical Block-Wise Neural Network Architecture Generation; May 2018;
    \url{https://arxiv.org/abs/1708.05552}

  \bibitem{raw18}
    Rawal, Miikkulainen - From Nodes to Networks: Evolving Recurrent Neural Networks; Jun 2018;
    \url{https://arxiv.org/abs/1803.04439}

  \bibitem{con18}
    Conti, Madhavan, et al - Improving Exploration in Evolution Strategies for Deep Reinforcement Learning via a Population of Novelty Seeking Agents; Oct 2018;
    \url{https://arxiv.org/abs/1712.06560}

  \bibitem{con18}
    Frolov - Neuroevolution: A Primer on Evolving Artificial Neural Networks; Oct 2018;
    \url{https://www.inovex.de/blog/neuroevolution/}

  \bibitem{rea19}
    Real, Aggarwal, et al - Regularized Evolution for Image Classifier Architecture Search; Feb 2019;
    \url{https://arxiv.org/abs/1802.01548}

  \bibitem{sun19}
    Sun, Xue, et al - Evolving Deep Convolutional Neural Networks for Image Classification; Mar 2019;
    \url{https://arxiv.org/abs/1710.10741}



\end{thebibliography}

\end{document}
